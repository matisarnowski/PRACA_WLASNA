\documentclass[a4paper,11pt]{article}
\usepackage[utf8]{inputenc}
\usepackage{latexsym}
\author{M. Sarnowski}
\title{Treść zadania.}
\frenchspacing
\begin{document}
\maketitle
\begin{flushleft}
Liczba $e$
\newline
Liczba $e$, to podstawa logarytmu naturalnego wykorzystywana w wielu dziedzinach matematyki i fizyki. W przybliżeniu wynosi 2,718281828459.
\newline
\newline
Zwana jest też liczbą Eulera lub liczbą Nepera.
\newline
\newline
Oblicz przybliżoną wartość liczby $e$, stosując szereg:
\newline
\newline
$e = \frac{1}{0!} + \frac{1}{1!} + \frac{1}{2!} + \frac{1}{3!} + \frac{1}{4!} + ... $
\newline
\end{flushleft}
\end{document}