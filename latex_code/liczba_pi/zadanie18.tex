\documentclass[a4paper,11pt]{article}
\usepackage[utf8]{inputenc}
\usepackage{latexsym}
\author{M. Sarnowski}
\title{Treść zadania.}
\frenchspacing
\begin{document}
\maketitle
Liczba $\pi$
Liczba $\pi$, inaczej ludolfina (od imienia holenderskiego matematyka Ludolfa van Ceulena), określa stosunek długości okręgu do długości jego średnicy.)
\newline
\newline
$\pi$ $\approx$ 3,141592 653589 793238 462643 383279 502884 197169 399375 105820 974944 592307 816406 286208 998628 034825 342117 067982 148086 513282 306647 093844 609550 582231 725359 408128 481117 450284 102701 938521 105559 644622 948954 930381 964428
\newline
\newline
Oblicz przybliżoną wartość liczby $\pi$, stosując szereg Leibniza:
\newline
\newline
$\frac{\pi}{4} = 1 - \frac{1}{3} + \frac{1}{5} - \frac{1}{7} + \frac{1}{9} -\frac{1}{11}... $
\newline
\end{document}