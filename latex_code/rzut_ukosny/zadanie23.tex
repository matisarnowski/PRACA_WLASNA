\documentclass[a4paper,11pt]{article}
\usepackage[utf8]{inputenc}
\author{M. Sarnowski}
\title{Rzut ukośny}
\frenchspacing
\begin{document}
\maketitle 
\begin{flushleft}
Rzut ukośny to ruch ciała, któremu nadano prędkość o wektorze skierowanym pod pewnym kątem do poziomu. Zakładamy, że ruch ten odbywa się bez żadnych oporów, np powietrza.
\newline
\newline
Dana jest odległość $(d)$ w metrach.
\newline
\newline
Oblicz, jaką trzeba nadać prędkość początkową, aby przedmiot osiągnął podaną odległość w poziomie dla kątów $1^{\circ}, 2^{\circ}, 3^{\circ}, ... 88^{\circ}, 89^{\circ}$.
\newline
\newline
Wzór na zasięg rzutu ukośnego: 
\begin{Large}
$\textit{z} = \frac{v^{2}_{0} \cdot \sin 2 \alpha}{\textit{g}}$
\end{Large}
\newline
\newline
$v_{0}$ to prędkość początkowa, $\alpha$ - kąt rzutu, $\textit{g} \approx 9,81 m/s^{2}$ - przyśpieszenie ziemskie.
\end{flushleft}
\end{document}