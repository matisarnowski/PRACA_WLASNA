\documentclass[a4paper,11pt]{article}
\usepackage[utf8]{inputenc}
\author{M. Sarnowski}
\title{Rozwiązywanie trójkąta}
\frenchspacing
\begin{document}
\maketitle
Dane są 3 liczby dodatnie. Sprawdź, czy mogą być one długościami boków trójkąta, a jeśli tak, to rozwiąż taki trójkąt. Rozwiązywanie trójkąta polega na podaniu wszystkich informacji o nim:
\begin{itemize}
\item obwód
\item pole (ze wzoru Herona)
\item kąty (z twierdzenia cosinusów)
\item rodzaj trójkąta
\end{itemize}
Miary kątów można obliczyć, stosując twierdzenie cosinusów.
\newline
$c^{2} = a^{2} + b^{2} - 2ab \cdot cos \gamma$
\newline
\newline
$2ab \cdot cos \gamma = a^{2} + b^{2} - c^{2}$
\newline
\newline
$\gamma = arccos \left( \frac{a^{2} + b^{2} - c^{2}}{2ab} \right) $
\newline
\newline
$\frac{\pi}{4} = 1 - \frac{1}{3} + \frac{1}{5} - \frac{1}{7} + \frac{1}{9} -\frac{1}{11}... $
\newline
\newline
Wzór Herona: $Pole = \sqrt{p \left( p - a \right) \left( p - b \right) \left( p - c \right)}$, gdzie $p$ to połowa obwodu trójkąta.
\newline
\newline
\begin{flushleft}
Należy pamiętać, że w językach programowania wszystkie funkcje trygonometryczne operują na miarach kątów wyrażonych w radianach.
\newline
Rodzaj trójkąta rozpoznaj po długościach boków i po miarach kątów.
\end{flushleft}
\end{document}